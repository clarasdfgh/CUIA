\chapter{Previa ejecución}
En este manual de usuario le guiaremos paso a paso en el uso de ARPi. Léalo cuidadosamente y en su enteridad antes de 
ejecutar el programa 

\section{Requisitos mínimos}
Estos son los elementos mínimos necesarios para la ejecución del programa:
\begin{itemize}
	\item Software: 
	\begin{itemize}
		\item Unity, versión 2020.3.30 o superior.
		\item Vuforia Unity Package, provisto con la práctica.
		\item Paquete ARpi, provisto con la práctica.
	\end{itemize}
	\item Hardware:
	\begin{itemize}
		\item Cámara web
		\item Micrófono
		\item Marcador para Vuforia, en nuestro caso una carta del juego de mesa Uno
	\end{itemize}
\end{itemize}

Con estos elementos podrá \textit{testear} las distintas funcionalidades del programa, si bien no podrá 
hacer uso de ellas adecuadamente.

\section{Requisitos recomendados}
Para hacer uso de la funcionalidad completa del programa, asegúrese de contar con los siguientes elementos:
\begin{itemize}
	\item Software: 
	\begin{itemize}
		\item Unity, versión 2020.3.30 o superior.
		\item Vuforia Unity Package, provisto con la práctica.
		\item Paquete ARpi, provisto con la práctica.
		\item En caso de usar dispositivo móvil: DroidCam, tanto en su dispositivo móvil, como en su ordenador.
	\end{itemize}
	\item Hardware:
	\begin{itemize}
		\item Cámara web y micrófono; O teléfono móvil.
		\item Marcador para Vuforia, en nuestro caso una carta del juego de mesa Uno.
		\item Soporte ajustable teléfono.
		\item Trípode y proyector.
		\item Piano o teclado.
	\end{itemize}
\end{itemize}

\subsubsection{Montaje}
En caso de disponer de todos los elementos necesarios para una ejecución completa, siga los siguientes pasos. En caso contrario, 
puede continuar en la siguiente sección.

\begin{enumerate}
	\item Conecte su dispositivo móvil mediante DroidCam a su ordenador. Puede hacerlo mediante cable USB o wi-fi.
	\item Ajuste el soporte del teléfono de manera que la cámara de este apunte al teclado. Asegúrese de que se enfoca correctamente 
	y que el teclado aparece entero.
	\item Conecte y configure el proyector. Móntelo en su trípode, que debe quedar a una altura mínima de aproximadamente de 1.5 metros.
	\item Posicione el trípode con el proyector inmediatamente detrás del asiento en el que se sentará para tocar. El trípode debe tener 
	suficiente altura como para que la proyección no se vea ofuscada por la persona sentada.  
	\item Desde la pantalla de calibración, ajuste la posición del proyector para que el piano real y el piano proyectado coincidan. 
\end{enumerate}

