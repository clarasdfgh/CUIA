\section{Reconocimiento del piano}
\subsection{Instalación de Vuforia}\label{vuforia}
Como hemos mencionado, la herramienta que decidimos usar para la realidad aumentada
en este proyecto es Vuforia, un software de pago que ofrece licencias gratuitas bajo
una serie de condiciones:
\begin{itemize}
	\item 20 model targets
	\item 20 area targets
	\item No superar las 100 imágenes en los servicios en la nube
\end{itemize}

Para poder usar la versión gratis debemos crearnos una cuenta en el portal de Vuforia
y crear en el apartado developing una licencia desde \textit{Develop}. Además
necesitamos importar la biblioteca a Unity, podemos descargarla desde la página oficial
y un asistente nos indicará los pasos en el entorno de desarrollo de Unity.
Cuando creemos nuestro primer objeto Vuforia, como \texttt{ARCamera}, podremos
acceder a su configuración y añadir nuestra licencia. 

\subsection{Algoritmo de reconocimiento}
El algoritmo de reconocimiento del teclado actúa de esta forma:
\begin{enumerate}
	\item Colocar un marcador aproximadamente en el centro del teclado
	\item Al ser reconocido se mostrará un modelo 3D de un teclado
	\item Calibraremos el teclado para colocarlo encima del nuestro y ajustarlo a su medida
	\item Confirmamos que la calibración está hecha y el modelo desaparece de la pantalla
\end{enumerate}
De esta forma no tenemos que gestionar muchos marcadores de reducido tamaño y será más
 eficiente y fácil de reconocer.

\subsubsection{Marcador}
El marcador es una imagen que se asocia al objeto \texttt{ImageTarget} que nos ofrece Vuforia,
le debemos asignar el modelo del piano creando un nodo hijo en unity con el. Sin mayor configuración,
al encender el simulador de programa y enfocar con la cámara se nos muestra el modelo
justo encima de nuestro marcador.


\subsubsection{Cámara}
Cuando creamos una nueva escena aparece con dos componentes: una cámara y un foco
de luz; para la realidad aumentada el primero de ellos no nos sirve, y hay que sustituirlo
por la \texttt{ARCamera} que nos ofrece Vuforia.

Para que use nuestra webcam debemos configurar el \texttt{PlayMode} en la configuración
del paquete.

Además, especificaremos un poco más el comportamiento del marcador y la \texttt{ARCamera} creando un script 
asociado a esta última con dos eventos personalizados de Unity que indican qué hará cuando encuente o pierda el marcador.
Este script es solo una modificación del ejemplo que viene el el pack de realidad aumentada de Vuforia para Unity, cuyo único añadido es 
mostrar un aviso de recalibración.

\subsubsection{Ajuste del modelo}
Para ajustar el modelo 3D a nuestro piano real, hemos usado el paquete \texttt{LeanTouch}
en su versión gratuita de la \textit{Unity Asset Store}. Esto nos proporcionará una
herramienta para redimensionar y mover el modelo que, aunque esté pensada para pantallas
táctiles, incluye también una forma de simular la interacción táctil en el ordenador pulsando la tecla \texttt{ctrl izquierdo}.

\section{Reconocimiento del audio}\label{audio}
\subsection{Captura del micrófono en tiempo real}
Para la captura, reproducción y análisis en tiempo real del audio, hemos hecho uso de
las clases \texttt{Microphone} y \texttt{AudioClip} de Unity. Creando una entidad de tipo
\texttt{AudioSource}, podemos hacer que su contenido sea guardado en un \texttt{AudioClip}, que
extraemos de un dispositivo \texttt{Microphone}.

\subsection{Procesamiento del audio y notación musical}
Para el desarrollo de esta funcionalidad, nos hemos apoyado en la biblioteca \\
\texttt{ExtractDataFromCapturedSoundByMicrophone} (Jorge Gutiérrez Segovia, 2021).
Cuando extraemos un clip de audio, podemos emplear el método
\texttt{GetData} de \texttt{AudioClip} para obtener información sobre, entre otras cosas, su frecuencia. La
información del sonido se reparte en un espectrograma de n elementos con valores
del 0 al 1, dependiendo de la amplitud relativa a el tramo de la frecuencia
dado.

Con este espectrograma, podemos extraer la frecuencia máxima relativa en un momento dado, y la
podemos correlacionar con la frecuencia de una nota musical conocida. La3 tiene una frecuencia de 440Hz,
y es el estándar de afinación mundial. Usando los 12 semitonos existentes (siete notas musicales separadas
por dos semitonos, menos Mi-Fa y Si-Do, que se separan por uno solo), podemos calcular la diferencia
con el La3 y obtener una nota y octava a partir de una frecuencia en herzios. Todo este
procesamiento lo lleva a cabo la función \texttt{GetSpectrumData} de la biblioteca anteriormente mencionada.
