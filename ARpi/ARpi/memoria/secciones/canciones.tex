\section{Lectura de partituras}
\subsection{MXML}
MXML es un formato estándar de partitura basado en XML, un lenguaje de marcado. Este formato es reconocido
por los principales software de edición y lectura de partituras digitales, como son MuseScore, Sibelius, Finale...

Su versatilidad propia de los lenguajes de marcado nos permite añadir una gran variedad de notaciones y adornos. Para este proyecto,
nos interesa la lectura secuencial de una voz - En versiones más avanzadas del programa, la lectura y representación de una segunda mano 
sería trivial, con el único detrimento del reconocimiento correcto del sonido en acordes. 

Tenemos la oportunidad de añadir partituras personalizadas con el uso de MXML: a partir de un archivo en este formato, no es 
difícil la generación de una versión simplificada con los elementos que nos atañen (nota y duración). Conocemos que su generación es 
implementable mediante scripts de macros en VIM, y de hecho pudimos desarrollar una versión inicial de un script con ayuda de
nuestro compañero y amigo Atanasio\ref{tacho}.

Nuestras partituras simplificadas XML siguen el siguiente esquema:

\begin{lstlisting}[language=xml]
    <song>
        <note>
            <pitch></pitch>
            <duration></duration>
        </note>
    </song>
\end{lstlisting} 

En esta versión del programa se incluyen unas demos de varias canciones populares:
\begin{enumerate} 
	\item Cantina Band - John Williams (Star Wars)
    \item The Entertainer - Scott Joplin
    \item Super Mario Theme - Koji Kondo (Super Mario Bros)
    \item Megalovania - Toby Fox (Undertale)
    \item Dirtmouth - Christopher Larkin (Hollow Knight)
\end{enumerate}