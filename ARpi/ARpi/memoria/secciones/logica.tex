\section{Implementación de la lógica de juego}

La lógica de nuestro programa es, en esencia, una serie de cubos de Unity cuya posición en el eje Y varía
descendientemente de forma constante a lo largo de la ejecución; y cuya altura se corresponde con la
duración de la nota. Su posición en el eje X dependerá de la calibración previamente realizada
con el modelo 3D del piano, del cual obtiene el cubo su ancho.

\subsection{PianoObject}
\texttt{PianoObject} es el comportamiento asociado al nodo que contiene el modelo 3D del piano. Está pensada
como una clase \textit{singleton}, de la cual sólo puede existir una instancia. Sus métodos más relevantes son los siguientes:
\begin{itemize}
	\item \texttt{GetChildren(int i)}: este método devuelve un objeto del conjunto del modelo del piano; es decir, 
	un \texttt{GameObject} que contiene todas las partes que componen una tecla, además del cubo que será la representación
	gráfica en pantalla de la nota.

	\item \texttt{GetNoteObject(int i)}: devuelve la representación gráfica de la nota. 
\end{itemize}

\subsection{NoteObject}
Esta clase se encarga de manejar las colisiones, que usamos para conocer el momento en el que se debe tocar una nota. 

Los Objetos de Unity cuentan con un colisionador por defecto \texttt{BoxCollider}, del cual haremos uso para determinar
el momento en el que la nota debe de ser tocada, y en tal caso, compararla con la nota esperada. Para implementarlo, hemos indicado
que la representación gráfica de las notas es un \textit{Trigger}, y además tenemos un plano con una etiqueta específica de colisionador.

Hemos sobrecargado los métodos asociados al colisionador por \textit{Trigger}:
\begin{itemize}
	\item \texttt{OnTriggerEnter(Collider other)}: método de evento para cuando entra por primera vez en contacto la nota
	con el plano colisionador. Activa un atributo privado (\texttt{can\_be\_pressed}).
	\item \texttt{OnTriggerStay(Collider other)}: mientras esté en contacto con el plano colisionador,
	llama a la función que compara la nota reconocida y la nota esperada. En
	caso de coincidir, añade 1 al contador de notas correctas.
	\item \texttt{OnTriggerExit(Collider other)}: método de evento para cuando la nota abandona el área del colisionador. 
	Desactiva el atributo privado (\texttt{can\_be\_pressed}) y, si no se ha reconocido ninguna nota, añade un error al contador de fallos.
\end{itemize}

\subsection{BeatScroller}
\texttt{BeatScroller} se encarga de la representación gráfica de la canción sobre la pantalla, así como de recorrer la canción.
Alguna de la información que contiene es:
\begin{itemize}
	\item Lista estática con los nombres de las notas existentes 
	\item Lista contenedora de los \texttt{NoteObject}
	\item Lista contenedora de las notas de la canción de forma secuencial
\end{itemize}

Sus métodos más relevantes son:
\begin{itemize} 
	\item \texttt{PlaySong(List<Note> notes)}: recibe la lista de objetos de la clase Nota y añade la representación gráfica
	correspondiente a la lista de \texttt{NoteObject} y los ordena.
	\item \texttt{AddChildren(string pitch)}: busca en el el \texttt{PianoObject} el objeto cuyo nombre corresponde al indicado.
	Crea una copia de la representación gráfia de la nota, la activa y la añade a la lista. En caso de recibir como argumento un 
	silencio, añade una nota de igual manera pero inactiva.
	\item \texttt{OrderNotes()}: modifica la altura en base a la duración asociada a la Nota, y la posición de forma relativa a la
	posición de la nota anterior.
	\item \texttt{CompareNotes()}: si el juego ha empezado, obtiene la información del micrófono tras procesarla como se explica 
	en la sección 2.2\ref{audio}. De esta información nos interesa el \textit{pitch}, el cual compara con el correspondiente
	en la lista de nombres, y devuelve si la nota escuchada se corresponde con la nota esperada, con margen de error de un semitono.
\end{itemize}

\subsection{SongSelector}
Esta clase se encarga de inicializar la lista de canciones. Obtiene todas las canciones del directorio Music, 
añade un evento \textit{onClick} desde la lista de botones, el cual
crea un objeto \texttt{Song}, cuyo funcionamiento describimos a continuación.

\subsubsection{Song}
Tiene una lista de objetos \texttt{Note}. Su constructor recibe la ruta a un documento XML del cual extraemos la información con 
funcionalidades de Unity para este tipo de archivo. Por cada nodo que encuentra, crea un objeto Note y lo añade a la lista.

\subsubsection{Note}
Objeto unitario que compone las canciones. Cuenta con pitch y duration.

\pagebreak
