\section{Implementación de los menús}
La aplicación consta de dos escenas:
\begin{itemize}
	\item Menú principal.
	\begin{itemize}
	\item Play
	\item Salir
	\end{itemize}

	\item Escena de canción.
	\begin{itemize}
	\item Calibración 3D
	\item Selector de canciones
	\item Juego
	\item Resultados
	\item Salida al menú principal
	\end{itemize}
\end{itemize}
El cambio entre escenas se realiza mediante la función \texttt{LoadScene} de la
clase \texttt{SceneManager} que nos ofrece Unity.
El script se asocia a un objeto que no se destruye entre escenas, ya que tiene
la orden \texttt{DontDestroyOnLoad} porque en Unity los objetos se destruyen al cambio
de escena, si no lo indicamos de forma explícita tendríamos que tener una copia del objeto
en cada escena o dejaría de funcionar el programa.

La escena para el menú consta de un \texttt{canvas} y varios \texttt{botones} (\textit{Play},
y \textit{Salir}). En esta versión del programa solo hay dos escenas, la calibración se realiza justo antes
de escoger la canción que se desea tocar.

\subsection{Escena para tocar}
Después de confirmar la calibración, se mostrarán 6 botones, cada uno de ellos con la demo de una canción.
Inmediatamente después comenzará la ejecución del juego.

\pagebreak
